\documentclass{article}

% ############### %
% PACKAGE IMPORTS %
% ############### %
\usepackage[utf8]{inputenc}
%\usepackage[margin=80pt]{geometry}
\usepackage{fancyhdr}
\usepackage{graphicx}
\usepackage[hidelinks]{hyperref}
\usepackage[most]{tcolorbox}
\usepackage{tikz}
\usetikzlibrary{shadows}

\pagestyle{fancy}

% ################ %
% ENVIRONMENT DEFS %
% ################ %
\newenvironment{guide}[1]
{
	\begin{center}
		\begin{tcolorbox}[%
			colback=black!20, 
			boxrule=0pt, 
			title=Step-by-step: #1,
			enhanced,
			breakable,
			overlay unbroken={%
                \draw[line width=1pt, black, rounded corners]
        	    (frame.north west) rectangle (frame.south east);
			},
    		overlay first={%
        		 \draw[line width=1pt, black, rounded corners]
        	    (frame.south west) -- (frame.north west) -- (frame.north east) -- (frame.south east);
                \draw[line width=1pt, black]
                (frame.south west) -- (frame.south east);
            },
    		overlay middle={%
                \draw[line width=1pt, black]
        	    (frame.north west) rectangle (frame.south east);
        	},
    		overlay last={%
                \draw[line width=1pt, black, rounded corners]
        	    (frame.north west) -- (frame.south west) -- (frame.south east) -- (frame.north east);
                \draw[line width=1pt, black]
                (frame.north west) -- (frame.north east);
           	}
        ]{}
    	\begin{enumerate}
}
{
    		\end{enumerate}
    	\end{tcolorbox}
	\end{center}  	 
}

\newcommand{\guideimage}[1]
{
	\begin{center}
		\includegraphics[width=0.5\textwidth]{#1}
	\end{center}
}


% Adapted from this StackExchange post:
% https://tex.stackexchange.com/a/5227
\newcommand*\keystroke[1]
{
	\raisebox{-1.5pt}
	{
		\hspace{-8pt}
		\begin{tikzpicture}
		\node
		[
			draw,
			fill=black!65,
			drop shadow={shadow xshift=0.25ex,shadow yshift=-0.25ex,fill=black,opacity=0.75},
      		rectangle,
      		rounded corners=2pt,
      		inner sep=3pt,
      		outer sep=0pt,
      		line width=0.5pt,
      		font=\scriptsize\sffamily,
      		text=black!10
    	]
    	{
    		\hspace*{0.5pt}\tt #1\hspace*{0.5pt}
    	}
    	;
    	\end{tikzpicture}
		\hspace{-8pt}
  	}
}

% ############## %
% VARIABLE SETUP %
% ############## %
\title{Blender Tools for GFS Documentation}
\author{Pherakki}
\date{v0.1}


% ############## %
% DOCUMENT START %
% ############## %
\begin{document}
\maketitle
\pagenumbering{gobble}
\clearpage

\tableofcontents
\clearpage

\pagenumbering{arabic} 

\section{Getting Started}
BlenderToolsForGFS is a \textbf{Blender 2.81+} plugin. It should work on all versions of Blender including and beyond 2.81. The plugin has been developed on versions 2.83 and 3.4.1.

\subsection{Installing the Plugin}


\subsection{Removing the Plugin}

\subsection{A Note on Expectations}
This plugin should be viewed as a \textbf{supplement} to existing model-editing tools, and not as a replacement. You should, in particular, post-process any models exported from Blender in \textbf{GFD Studio}, following existing tutorials and knowledge to achieve your goals.

Please also note that although this document does offer a number of guides on bits of basic Blender usage required to use the plugin, this is not a guide on how to use Blender. If you are intending to work with models and animations, it is ultimately your responsibility to learn how to use Blender using the wealth of resources available on the internet.

This is also not a guide on how to mod games. Again, it is your responsibility to seek out or discover the information you require to attain any such goals.

What this guide \textit{is} intended for is to assist you in successfully exporting data from Blender to the GFS file format. Suggestions for improvements on how to do that are very much welcomed, as long as they do not fall out-of-scope for the essential use of the plugin.
\clearpage

\section{Import}
\subsection{Import Restrictions}
Currently, a subset of features of the GFS format are not supported by the plugin and cannot be exported. These are:
\begin{itemize}
\item File versions outside of the range 0x01104920 - 0x01105100
\item EPL Data (Subfiles / Particle Effects)
\item Morphs / Shapekeys
\item Strings that are not SHIFT-JIS- (for texture file names) or UTF8- (everything else) encoded
\end{itemize}

There are also some additional considerations to bear in mind.
\begin{itemize}
\item During import, meshes are re-parented from their positioning node to the root node, whilst preserving their world transform. This is because there does not seem to be a straightforward way to make a mesh follow the transform of a bone whilst also being deformed by an armature without Blender double-counting transformations. These positioning nodes are also removed from the import if they are not used for any other purpose.
\item Animations of the Root Node and of mesh positioning nodes are not editable from Blender. Any mesh positioning node animations \textbf{will be messed up} by the Blender import if these nodes were not originally children of the root node. This must be fixed outside of Blender by re-parenting and repositioning these nodes.
\item Only non-BC7 DDS texture slots are currently importable in models.
\end{itemize}

\clearpage 
\subsection{Importing Model (GMD/GFS) Files}
\begin{guide}{Accessing the Model Import Menu}
\item Open the File menu and navigate to the \textbf{Import} \textgreater\space \textbf{GFS} \textgreater\space \textbf{GFS Model} submenu item.
\guideimage{images/import/import_gmd.png}
\item Select the GMD or GFS file to import using the file browser that pops up.
\end{guide}


\subsection{Importing Animation (GAP) Files}
\begin{guide}{Accessing the Animation Import Menu}
\item Ensure that you have first imported the model that the animation is modelled for.
\item Open the File menu and navigate to the \textbf{Import} \textgreater\space \textbf{GFS} \textgreater\space \textbf{GAP Animation} submenu item.
\guideimage{images/import/import_gap.png}
\item Select the armature for your imported model using the drop-down.
\guideimage{images/import/import_gap_armature_select.png}
\item Select the GAP file to import using the file browser that pops up.
\end{guide}

\subsection{Blender Settings}
\subsubsection{Previewing Materials}
Many beginners to Blender are confused by the fact that models do not display their materials when it is first opened. Blender by default renders models in a ``Solid" representation that is useful for modellers editing meshes. You can preview materials instead by changing this render setting.
\begin{guide}{Previewing Materials}
\item Locate the \textbf{Viewport Shading} widget and select \textbf{Material Preview} mode.
\guideimage{images/import/import_preview_materials.png}
\end{guide}

\subsubsection{Importing Large Models}
\label{SECTION::ImportingLargeModels}
Some models, such as Field Models, are typically in such large units that they exceed the default render distance of Blender. In this instance, you may find it useful to increase Blender's render distance.\\
\begin{guide}{Increasing Blender's Render Distance}
\item Press the \keystroke{N} key to open the \textbf{Sidebar}.
\guideimage{images/import/import_field_open_menu.png}
\item Click the \textbf{View} tab in the \textbf{Sidebar}.
\guideimage{images/import/import_field_open_view.png}
\item Change the value in the \textbf{End} box to a value large enough for you to comfortably navigate the model.
\guideimage{images/import/import_field_set_distance.png}
\item Press the \keystroke{N} key to close the \textbf{Sidebar}.
\end{guide}

\subsubsection{Hiding Unused Bones}
On import, three bone layers will be created in the armature:
\begin{itemize}
\item All Bones
\item Bones used in Vertex Groups
\item Bones not used in Vertex Groups
\end{itemize}
Selecting one of these layers will allow you to hide these subsets of bones. Note that these groups are created \textbf{by the plugin} on import, and if you add new bones to a model then it is up to you to add bones to whatever bone layers you want.

\begin{guide}{Selecting Bone Layers}
\item Switch to Object Mode.
\guideimage{images/import/import_to_object_mode.png}
\item Select the model armature.
\item Click the Armature icon in the Properties Panel. Select the Bone Layer that is active in the Viewport.
\guideimage{images/import/import_bone_layer_select.png}
\end{guide}

\clearpage

\section{Editing Models and Animations}
\subsection{GFS Models}
Models are imported as Armature objects with meshes, cameras, and lights parented below them. Most of the objects imported from GFS files can carry additional information beyond pure geometry and shading data, which is outlined in the proceeding sections.

GFS models are very large compared to typical Blender scales. If parts of the model disappear as you zoom out, you will need to change the render distance of Blender as described in section \ref{SECTION::ImportingLargeModels}.

The data in the files are assumed to be oriented Y-up and with X-axis-oriented bones. During import, these are converted to the Z-up orientation and Y-axis-oriented bones to match the Blender conventions. On export, the Blender data is converted back to Y-up orientation and X-axis-oriented bones. 

\subsubsection{Bones}
\label{SECTION::EditingBones} 
There are no special considerations for bones beyond how they behave in Blender. However, bones do also have some additional information that can be attached to them not representable in Blender. These can be accessed from the \textbf{GFS Node} panel in the \textbf{Bone Properties}.

\begin{itemize}
\item \textbf{Unknown Float}: Unknown. Always seems to take a value of 1.
\item \textbf{Properties}: Custom properties attached to the bone. See section \ref{SECTION::GfsProperties} for further details.
\end{itemize}

\begin{guide}{Accessing Extra Bone Properties}
\item Select the armature of the model either in the Viewport or in the Outliner.
\guideimage{images/editing_models/edits_select_armature.png}
\item Switch to Pose Mode.
\guideimage{images/editing_models/edits_to_pose_mode.png}
\item Select a bone and select the Bone icon in the Properties panel. You will find a panel called ``GFS Node" containing the additional Bone Properties.
\guideimage{images/editing_models/edits_bone_properties.png}
\end{guide}
\clearpage

\subsubsection{Rest Pose}

\subsubsection{Meshes}
Meshes that belong to a GFS model are parented under an armature. Meshes can use an armature deform with vertex groups as usual inside Blender. However, each vertex is only allowed to be part of a maximum of 4 vertex groups, and a maximum of 256 vertex groups are permitted across the entire model. If two meshes are weighted to the same bone, this counts as two vertex groups, since vertex groups are tied to the transform of a mesh in the GFS file format. The two vertex groups can be counted as one if meshes are parented to each other, as will be described momentarily.

The transform of the mesh inside Blender is exported as the mesh transform within the GFS files. Note however that meshes can be parented under other meshes to ensure that they share a transform. In this situation, the transform of the child-mesh is ignored and should always be a unit transform if you want to accurately see what will be exported in the viewport. In addition, if a mesh and its parent-mesh both use the same bone as a vertex group, it will be counted once instead of double-counted since both meshes have the same transform. Only one level of mesh-to-mesh parenting will be detected on export.

\begin{guide}{Mesh Parenting for GFS}
\item Meshes must be parented under an armature. The transform of a mesh dictates the transform of the mesh in the exported file.
\guideimage{images/editing_models/edits_parent_mesh.png}
\item Meshes can also be parented under other meshes. The transform of these child meshes will be ignored. Parenting a mesh under another mesh ensures that they will share a transform in the exported file and vertex groups used by both meshes will not be double-counted by the file format.
\guideimage{images/editing_models/edits_child_mesh.png}
\end{guide}

\begin{guide}{Unparenting Objects within Blender}
\item Select an object in the outliner or in the viewport.
\item Press \keystroke{Alt} + \keystroke{P} with the mouse in the Viewport.
\item Select either:
\begin{enumerate}
\item \textbf{Clear Parent} if you want to unparent the object from its parent \textbf{and} move the mesh so that its transform is relative to the origin.
\item \textbf{Clear and Keep Transform} if you want to unparent the object from its parent \textbf{and} edit the mesh's transform so that it stays in the same place in the viewport.
\end{enumerate}
Ignore \textbf{Clear Parent Inverse}.
\guideimage{images/editing_models/editing_unparent.png}
\end{guide}

\begin{guide}{Parenting Objects within Blender}
\item Select the child object in the outliner or in the viewport.
\guideimage{images/editing_models/edits_parent_1.png}
\item Select the parent object in the outliner or in the viewport with \keystroke{Ctrl} + Click.
\guideimage{images/editing_models/edits_parent_2.png}
\item Press \keystroke{Ctrl} + \keystroke{P} with the mouse in the Viewport.
\item Select either:
\begin{enumerate}
\item \textbf{Object (Without Inverse)} if you want to parent the object and reset its transform.
\item \textbf{Object (Keep Transform Without Inverse)} if you want to parent the object \textbf{and} edit the mesh's transform so that it stays in the same place in the viewport.
\end{enumerate}
\guideimage{images/editing_models/edits_parent_3.png}
\item In any situation, if after parenting your child object's transform is not what you expected, then:
\begin{itemize}
\item Select the child object.
\item Press \keystroke{Alt} + \keystroke{P} with the mouse in the Viewport.
\item Click \textbf{Clear Parent Inverse}.
\end{itemize}
This will remove the hidden ``parent inverse" matrix that sometimes gets inserted when parenting objects. This is harmless, but merely means that you object's transform may not align with what you see in the viewport if it is not a unit transform.
\guideimage{images/editing_models/editing_unparent.png}
\end{guide}

Meshes have certain attributes that may be exported. These are:
\begin{itemize}
\item \textbf{Export Bounding Box}/\textbf{Sphere}: Define a bounding Box / Sphere on export.
\item \textbf{Export Normals}/\textbf{Tangents}/\textbf{Binormals}: Export these attributes. They are required for specific material options to work.
\item \textbf{Unknown Flag}: The purpose of Unknown Flags is not known. Checking and unchecking these may cause or fix crashes.
\item \textbf{Unknown 0x12}: Purpose unknown.
\item The GFS Node sub-panel will appear if the Mesh is not parented under another Mesh. Refer to section \ref{SECTION::EditingBones} for further details.
\end{itemize}

\begin{guide}{Accessing Extra Mesh Attributes}
\item Select the mesh in the Outliner or in the Viewport.
\guideimage{images/editing_models/edits_select_mesh.png}
\item Switch to Object Mode.
\guideimage{images/editing_models/edits_to_object_mode.png}
\item Select the Mesh icon in the Properties panel. You will find a panel called ``GFS Mesh" containing the additional Mesh attributes.
\guideimage{images/editing_models/edits_mesh_properties.png}
\end{guide}

\underline{\textbf{WARNINGS}}
\begin{itemize}
\item Many tutorials and help articles will tell you to \textbf{Apply Transforms} to your mesh to reset their transforms to a unit transform. \textbf{This is dangerous}. When you \textbf{Apply Transforms}, you are translating, rotating, and scaling the vertices that make up the mesh, rather than applying an extra transform on top of the mesh. This means that, for example, your mesh will now use the origin as the reference point from which it rotates and scales, rather than the point around which the mesh was modelled.
\end{itemize}

\subsubsection{Materials}
\label{SECTION::EditingMaterials}

Materials are not yet sufficiently understood to a degree whereby they can be consistently rendered in Blender. Therefore, materials are only represented in Blender as the Diffuse Texture of the material and all other properties are inferred from attributes or the names of Shader Nodes. Edits to the GFS Material attributes will, more often than not, simply lead to the model crashing any game that loads it. Due to this, editing materials from Blender is \textbf{not} currently recommended, except for the purposes of:
\begin{enumerate}
\item Adding textures to a material.
\item Perhaps minor tweaks to attributes.
\item Researching what the attributes do.
\end{enumerate}
You will have much more success following the strategy of current model editing guides: post-processing the model in \textbf{GFD Studio} by copying materials from other models onto your exported model. How to do this is beyond the scope of this documentation but detailed tutorials are available online.

In the future, when materials are properly understood, a custom Shader Node Group should be implemented that allows the material to be rendered in a meaningful fashion, and allowing the values of attributes to be auto-calculated such that they do not crash the game. \textbf{There is not enough knowledge currently to do this and any contributions to material research enabling this feature is highly welcomed.}

There are then two essential pieces to Material editing:
\begin{itemize}
\item Texture Samplers
\item GFS Material Attributes
\end{itemize}

\noindent\underline{Texture Samplers}\\
You can edit the textures used by a material by accessing the Shader Nodes. The first step is to open the Shader Node Editor.
\begin{guide}{Opening the Shader Node Editor}
\item Select the mesh in the Outliner or in the Viewport.
\guideimage{images/editing_models/edits_select_mesh.png}
\item Switch to Object Mode.
\guideimage{images/editing_models/edits_to_object_mode.png}
\item Select the Material icon in the Properties panel.
\guideimage{images/editing_models/edits_select_material.png}
\item Select the Shader Editor.
\guideimage{images/editing_models/edits_open_shader_nodes.png}
\end{guide}

You can then set create new samplers to be exported and set their GFS attributes. There are nine types of textures available in the GFS format. Each material can have \textbf{one} of each type, and the plugin will recognise which type of sampler a node represents by the \textbf{name} of the node. The recognised names are:
\begin{itemize}
\item Diffuse Texture
\item Normal Texture
\item Specular Texture
\item Reflection Texture
\item Highlight Texture
\item Glow Texture
\item Detail Texture
\item Night Texture
\item Shadow Texture
\end{itemize}

Note also that the UV maps must also have specific names due to the way they are stored in the GFS file format. The permitted names are:
\begin{itemize}
\item UV0
\item UV1
\item UV2
\item UV3
\item UV4
\item UV5
\item UV6
\item UV7
\end{itemize}
meaning that up to 8 maps are allowed per mesh.

\begin{guide}{Editing Texture Samplers}
\item Inside the Shader Editor, select or create an Image Node.
\item Select the \textbf{Node} tab in the Sidebar. If the Sidebar is not open, press \keystroke{N} with the mouse inside the Shader Editor.
\item Set the name of the node to one of the nine accepted names by changing the value in the Name field.
\guideimage{images/editing_models/edits_rename_shader_node.png}
\item Select or create a UV Map node. Select a UV map present on the mesh from the drop-down on the node and hook the UV connector up to the Vector connector on the Texture node.
\guideimage{images/editing_models/edits_set_node_uv_map.png}
\item You can access the properties of a texture sampler by selecting the \textbf{GFS Texture} tab in the Sidebar. The properties for the sampler are given in the \textbf{GFS Texture Sampler} panel.
\guideimage{images/editing_models/edits_texture_sampler_properties.png}
\end{guide}

\noindent\underline{GFS Material Attributes}\\
\noindent Materials have certain attributes that may be exported. These are:
\begin{itemize}
\item \textbf{Unknown Flags}: The purpose of Unknown Flags is not known. Checking and unchecking these may cause or fix crashes.
\item \textbf{Enable Vertex Colors}: Use Color Map data on the mesh.
\item \textbf{Enable UV Anims}: Allow the UV coordinates of the mesh to be animated.
\item \textbf{Use Light 2}: Purpose unknown.
\item \textbf{Purple Wireframe}: Render the mesh as a purple wireframe.
\item The GFS Node sub-panel will appear if the Mesh is not parented under another Mesh. Refer to section \ref{SECTION::EditingBones} for further details.
\end{itemize}
\begin{guide}{Accessing Extra Material Attributes}
\item Select the mesh in the Outliner or in the Viewport.
\guideimage{images/editing_models/edits_select_mesh.png}
\item Switch to Object Mode.
\guideimage{images/editing_models/edits_to_object_mode.png}
\item Select the Material icon in the Properties panel. You will find a panel called ``GFS Material" containing the additional Material attributes.
\guideimage{images/editing_models/edits_material_properties.png}
\end{guide}

\subsubsection{Textures}
Textures must be DDS images with either no compression, or DXT1, DX3, or DXT5 compression. BC7 textures cannot currently be loaded by Blender and are currently unsupported. In the future, BC7 textures should be loadable \textit{via} an automatic conversion to a Blender-readable format.

\noindent Materials have certain attributes that may be exported. These are:
\begin{itemize}
\item \textbf{Unknown 1}: Purpose unknown.
\item \textbf{Unknown 2}: Purpose unknown.
\item \textbf{Unknown 3}: Purpose unknown.
\item \textbf{Unknown 4}: Purpose unknown.
\end{itemize}
These attributes can be edited from the \textbf{GFS Texture} Sidebar panel on a Texture Image node in the Shader Editor, as described in section \ref{SECTION::EditingMaterials}.

\subsubsection{Cameras}
Cameras are parented to bones on the model armature. The rest transform of a camera is:
\begin{itemize}
\item A 90 degree rotation in the Z axis
\item A Y translation that puts the camera at the head of the bone it is parented to.
\end{itemize}
The Camera will have a View Matrix defined relative to this transform.

\subsubsection{Lights}


\subsubsection{Physics}
Model physics are not currently editable from Blender.

\subsection{GFS Animations}

\subsubsection{Normal Animations}

\subsubsection{Blend Animations}

\subsubsection{LookAt Animations}

\subsection{GFS Properties}
\label{SECTION::GfsProperties}


\section{Export}
\subsection{Exporting Model (GMD/GFS) Files}
\subsection{Exporting Animation (GAP) Files}
\clearpage

\section{BlenderToolsForGFS as a Python Library}
\clearpage

\end{document}
