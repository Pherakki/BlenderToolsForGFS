\documentclass{article}

% ############### %
% PACKAGE IMPORTS %
% ############### %
\usepackage[utf8]{inputenc}
%\usepackage[margin=80pt]{geometry}
\usepackage{graphicx}
\usepackage[hidelinks]{hyperref}
\usepackage[most]{tcolorbox}
\usepackage{tikz}
\usetikzlibrary{shadows}

% ################ %
% ENVIRONMENT DEFS %
% ################ %
\newenvironment{guide}[1]
{
	\begin{center}
		\begin{tcolorbox}[%
			colback=black!20, 
			boxrule=0pt, 
			title=Step-by-step: #1,
			enhanced,
			breakable,
			overlay unbroken={%
                \draw[line width=1pt, black, rounded corners]
        	    (frame.north west) rectangle (frame.south east);
			},
    		overlay first={%
        		 \draw[line width=1pt, black, rounded corners]
        	    (frame.south west) -- (frame.north west) -- (frame.north east) -- (frame.south east);
                \draw[line width=1pt, black]
                (frame.south west) -- (frame.south east);
            },
    		overlay middle={%
                \draw[line width=1pt, black]
        	    (frame.north west) rectangle (frame.south east);
        	},
    		overlay last={%
                \draw[line width=1pt, black, rounded corners]
        	    (frame.north west) -- (frame.south west) -- (frame.south east) -- (frame.north east);
                \draw[line width=1pt, black]
                (frame.north west) -- (frame.north east);
           	}
        ]{}
    	\begin{enumerate}
}
{
    		\end{enumerate}
    	\end{tcolorbox}
	\end{center}  	 
}

\newcommand{\guideimage}[1]
{
	\begin{center}
		\includegraphics[width=0.5\textwidth]{#1}
	\end{center}
}


% Adapted from this StackExchange post:
% https://tex.stackexchange.com/a/5227
\newcommand*\keystroke[1]
{
	\raisebox{-1.5pt}
	{
		\hspace{-8pt}
		\begin{tikzpicture}
		\node
		[
			draw,
			fill=black!65,
			drop shadow={shadow xshift=0.25ex,shadow yshift=-0.25ex,fill=black,opacity=0.75},
      		rectangle,
      		rounded corners=2pt,
      		inner sep=3pt,
      		outer sep=0pt,
      		line width=0.5pt,
      		font=\scriptsize\sffamily,
      		text=black!10
    	]
    	{
    		\hspace*{0.5pt}\tt #1\hspace*{0.5pt}
    	}
    	;
    	\end{tikzpicture}
		\hspace{-8pt}
  	}
}

% ############## %
% VARIABLE SETUP %
% ############## %
\title{Blender Tools for GFS Documentation}
\author{Pherakki}
\date{v0.1}


% ############## %
% DOCUMENT START %
% ############## %
\begin{document}
\maketitle
\pagenumbering{gobble}
\clearpage

\tableofcontents
\clearpage

\pagenumbering{arabic} 

\section{Getting Started}
\clearpage

\section{Import}
\subsection{Importing Field Models}
Field models are typically in such large units that they exceed the default render distance of Blender. In this instance, you may find it useful to increase Blender's render distance.\\
\begin{guide}{Increasing Blender's Render Distance}
\item Press the \keystroke{N} key to open the \textbf{Sidebar}.
\guideimage{images/import/import_field_open_menu.png}
\item Click the \textbf{View} tab in the \textbf{Sidebar}.
\guideimage{images/import/import_field_open_view.png}
\item Change the value in the \textbf{End} box to a value large enough for you to comfortably navigate the model.
\guideimage{images/import/import_field_set_distance.png}
\item Press the \keystroke{N} key to close the \textbf{Sidebar}.
\end{guide}
\clearpage

\section{Manipulating Data}
\clearpage

\section{Export}
\subsection{Preparing Data for Export}
\clearpage

\section{BlenderToolsForGFS as a Python Library}
\clearpage

\end{document}
