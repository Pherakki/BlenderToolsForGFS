\documentclass{article}

% ############### %
% PACKAGE IMPORTS %
% ############### %
\usepackage[utf8]{inputenc}
%\usepackage[margin=80pt]{geometry}
\usepackage{graphicx}
\usepackage[hidelinks]{hyperref}
\usepackage[most]{tcolorbox}
\usepackage{tikz}
\usetikzlibrary{shadows}

% ################ %
% ENVIRONMENT DEFS %
% ################ %
\newenvironment{guide}[1]
{
	\begin{center}
		\begin{tcolorbox}[%
			colback=black!20, 
			boxrule=0pt, 
			title=Step-by-step: #1,
			enhanced,
			breakable,
			overlay unbroken={%
                \draw[line width=1pt, black, rounded corners]
        	    (frame.north west) rectangle (frame.south east);
			},
    		overlay first={%
        		 \draw[line width=1pt, black, rounded corners]
        	    (frame.south west) -- (frame.north west) -- (frame.north east) -- (frame.south east);
                \draw[line width=1pt, black]
                (frame.south west) -- (frame.south east);
            },
    		overlay middle={%
                \draw[line width=1pt, black]
        	    (frame.north west) rectangle (frame.south east);
        	},
    		overlay last={%
                \draw[line width=1pt, black, rounded corners]
        	    (frame.north west) -- (frame.south west) -- (frame.south east) -- (frame.north east);
                \draw[line width=1pt, black]
                (frame.north west) -- (frame.north east);
           	}
        ]{}
    	\begin{enumerate}
}
{
    		\end{enumerate}
    	\end{tcolorbox}
	\end{center}  	 
}

\newcommand{\guideimage}[1]
{
	\begin{center}
		\includegraphics[width=0.5\textwidth]{#1}
	\end{center}
}


% Adapted from this StackExchange post:
% https://tex.stackexchange.com/a/5227
\newcommand*\keystroke[1]
{
	\raisebox{-1.5pt}
	{
		\hspace{-8pt}
		\begin{tikzpicture}
		\node
		[
			draw,
			fill=black!65,
			drop shadow={shadow xshift=0.25ex,shadow yshift=-0.25ex,fill=black,opacity=0.75},
      		rectangle,
      		rounded corners=2pt,
      		inner sep=3pt,
      		outer sep=0pt,
      		line width=0.5pt,
      		font=\scriptsize\sffamily,
      		text=black!10
    	]
    	{
    		\hspace*{0.5pt}\tt #1\hspace*{0.5pt}
    	}
    	;
    	\end{tikzpicture}
		\hspace{-8pt}
  	}
}

% ############## %
% VARIABLE SETUP %
% ############## %
\title{Blender Tools for GFS Documentation}
\author{Pherakki}
\date{v0.1}


% ############## %
% DOCUMENT START %
% ############## %
\begin{document}
\maketitle
\pagenumbering{gobble}
\clearpage

\tableofcontents
\clearpage

\pagenumbering{arabic} 

%\section{Getting Started}
%\clearpage

\section{Import}
\subsection{Importing Model (GMD/GFS) Files}
\begin{guide}{Accessing the Model Import Menu}
\item Open the File menu and navigate to the \textbf{Import} \textgreater\space \textbf{GFS} \textgreater\space \textbf{GFS Model} submenu item.
\guideimage{images/import/import_gmd.png}
\item Select the GMD or GFS file to import using the file browser that pops up.
\end{guide}


\subsection{Importing Animation (GAP) Files}
\begin{guide}{Accessing the Animation Import Menu}
\item Ensure that you have first imported the model that the animation is modelled for.
\item Open the File menu and navigate to the \textbf{Import} \textgreater\space \textbf{GFS} \textgreater\space \textbf{GAP Animation} submenu item.
\guideimage{images/import/import_gap.png}
\item Select the armature for your imported model using the drop-down.
\guideimage{images/import/import_gap_armature_select.png}
\item Select the GAP file to import using the file browser that pops up.
\end{guide}

\subsection{Blender Settings}
\subsubsection{Previewing Materials}
Many beginners to Blender are confused by the fact that models do not display their materials when it is first opened. Blender by default renders models in a ``Solid" representation that is useful for modellers editing meshes. You can preview materials instead by changing this render setting.
\begin{guide}{Previewing Materials}
\item Locate the \textbf{Viewport Shading} widget and select \textbf{Material Preview} mode.
\guideimage{images/import/import_preview_materials.png}
\end{guide}

\subsubsection{Importing Large Models}
Some models, such as Field Models, are typically in such large units that they exceed the default render distance of Blender. In this instance, you may find it useful to increase Blender's render distance.\\
\begin{guide}{Increasing Blender's Render Distance}
\item Press the \keystroke{N} key to open the \textbf{Sidebar}.
\guideimage{images/import/import_field_open_menu.png}
\item Click the \textbf{View} tab in the \textbf{Sidebar}.
\guideimage{images/import/import_field_open_view.png}
\item Change the value in the \textbf{End} box to a value large enough for you to comfortably navigate the model.
\guideimage{images/import/import_field_set_distance.png}
\item Press the \keystroke{N} key to close the \textbf{Sidebar}.
\end{guide}
\clearpage

\section{Editing Models and Animations}
\subsection{GFS Models}


\subsubsection{Bones}
\begin{guide}{Accessing Extra Properties}
\item placeholder
\guideimage{images/editing_models/edits_bone_properties.png}
\end{guide}
\clearpage

\subsubsection{Rest Pose}

\subsubsection{Meshes}


\subsubsection{Materials}

\subsubsection{Textures}

\subsubsection{Cameras}

\subsubsection{Lights}

\subsubsection{Physics}
Model physics are not currently editable from Blender.

\subsection{GFS Animations}

\subsubsection{Normal Animations}

\subsubsection{Blend Animations}

\subsubsection{LookAt Animations}


\section{Export}
\subsection{Exporting Model (GMD/GFS) Files}
\subsection{Exporting Animation (GAP) Files}
\clearpage

\section{BlenderToolsForGFS as a Python Library}
\clearpage

\end{document}
